% Options for packages loaded elsewhere
\PassOptionsToPackage{unicode}{hyperref}
\PassOptionsToPackage{hyphens}{url}
%
\documentclass[
]{article}
\usepackage{amsmath,amssymb}
\usepackage{iftex}
\ifPDFTeX
  \usepackage[T1]{fontenc}
  \usepackage[utf8]{inputenc}
  \usepackage{textcomp} % provide euro and other symbols
\else % if luatex or xetex
  \usepackage{unicode-math} % this also loads fontspec
  \defaultfontfeatures{Scale=MatchLowercase}
  \defaultfontfeatures[\rmfamily]{Ligatures=TeX,Scale=1}
\fi
\usepackage{lmodern}
\ifPDFTeX\else
  % xetex/luatex font selection
\fi
% Use upquote if available, for straight quotes in verbatim environments
\IfFileExists{upquote.sty}{\usepackage{upquote}}{}
\IfFileExists{microtype.sty}{% use microtype if available
  \usepackage[]{microtype}
  \UseMicrotypeSet[protrusion]{basicmath} % disable protrusion for tt fonts
}{}
\makeatletter
\@ifundefined{KOMAClassName}{% if non-KOMA class
  \IfFileExists{parskip.sty}{%
    \usepackage{parskip}
  }{% else
    \setlength{\parindent}{0pt}
    \setlength{\parskip}{6pt plus 2pt minus 1pt}}
}{% if KOMA class
  \KOMAoptions{parskip=half}}
\makeatother
\usepackage{xcolor}
\usepackage[margin=1in]{geometry}
\usepackage{color}
\usepackage{fancyvrb}
\newcommand{\VerbBar}{|}
\newcommand{\VERB}{\Verb[commandchars=\\\{\}]}
\DefineVerbatimEnvironment{Highlighting}{Verbatim}{commandchars=\\\{\}}
% Add ',fontsize=\small' for more characters per line
\usepackage{framed}
\definecolor{shadecolor}{RGB}{248,248,248}
\newenvironment{Shaded}{\begin{snugshade}}{\end{snugshade}}
\newcommand{\AlertTok}[1]{\textcolor[rgb]{0.94,0.16,0.16}{#1}}
\newcommand{\AnnotationTok}[1]{\textcolor[rgb]{0.56,0.35,0.01}{\textbf{\textit{#1}}}}
\newcommand{\AttributeTok}[1]{\textcolor[rgb]{0.13,0.29,0.53}{#1}}
\newcommand{\BaseNTok}[1]{\textcolor[rgb]{0.00,0.00,0.81}{#1}}
\newcommand{\BuiltInTok}[1]{#1}
\newcommand{\CharTok}[1]{\textcolor[rgb]{0.31,0.60,0.02}{#1}}
\newcommand{\CommentTok}[1]{\textcolor[rgb]{0.56,0.35,0.01}{\textit{#1}}}
\newcommand{\CommentVarTok}[1]{\textcolor[rgb]{0.56,0.35,0.01}{\textbf{\textit{#1}}}}
\newcommand{\ConstantTok}[1]{\textcolor[rgb]{0.56,0.35,0.01}{#1}}
\newcommand{\ControlFlowTok}[1]{\textcolor[rgb]{0.13,0.29,0.53}{\textbf{#1}}}
\newcommand{\DataTypeTok}[1]{\textcolor[rgb]{0.13,0.29,0.53}{#1}}
\newcommand{\DecValTok}[1]{\textcolor[rgb]{0.00,0.00,0.81}{#1}}
\newcommand{\DocumentationTok}[1]{\textcolor[rgb]{0.56,0.35,0.01}{\textbf{\textit{#1}}}}
\newcommand{\ErrorTok}[1]{\textcolor[rgb]{0.64,0.00,0.00}{\textbf{#1}}}
\newcommand{\ExtensionTok}[1]{#1}
\newcommand{\FloatTok}[1]{\textcolor[rgb]{0.00,0.00,0.81}{#1}}
\newcommand{\FunctionTok}[1]{\textcolor[rgb]{0.13,0.29,0.53}{\textbf{#1}}}
\newcommand{\ImportTok}[1]{#1}
\newcommand{\InformationTok}[1]{\textcolor[rgb]{0.56,0.35,0.01}{\textbf{\textit{#1}}}}
\newcommand{\KeywordTok}[1]{\textcolor[rgb]{0.13,0.29,0.53}{\textbf{#1}}}
\newcommand{\NormalTok}[1]{#1}
\newcommand{\OperatorTok}[1]{\textcolor[rgb]{0.81,0.36,0.00}{\textbf{#1}}}
\newcommand{\OtherTok}[1]{\textcolor[rgb]{0.56,0.35,0.01}{#1}}
\newcommand{\PreprocessorTok}[1]{\textcolor[rgb]{0.56,0.35,0.01}{\textit{#1}}}
\newcommand{\RegionMarkerTok}[1]{#1}
\newcommand{\SpecialCharTok}[1]{\textcolor[rgb]{0.81,0.36,0.00}{\textbf{#1}}}
\newcommand{\SpecialStringTok}[1]{\textcolor[rgb]{0.31,0.60,0.02}{#1}}
\newcommand{\StringTok}[1]{\textcolor[rgb]{0.31,0.60,0.02}{#1}}
\newcommand{\VariableTok}[1]{\textcolor[rgb]{0.00,0.00,0.00}{#1}}
\newcommand{\VerbatimStringTok}[1]{\textcolor[rgb]{0.31,0.60,0.02}{#1}}
\newcommand{\WarningTok}[1]{\textcolor[rgb]{0.56,0.35,0.01}{\textbf{\textit{#1}}}}
\usepackage{graphicx}
\makeatletter
\def\maxwidth{\ifdim\Gin@nat@width>\linewidth\linewidth\else\Gin@nat@width\fi}
\def\maxheight{\ifdim\Gin@nat@height>\textheight\textheight\else\Gin@nat@height\fi}
\makeatother
% Scale images if necessary, so that they will not overflow the page
% margins by default, and it is still possible to overwrite the defaults
% using explicit options in \includegraphics[width, height, ...]{}
\setkeys{Gin}{width=\maxwidth,height=\maxheight,keepaspectratio}
% Set default figure placement to htbp
\makeatletter
\def\fps@figure{htbp}
\makeatother
\setlength{\emergencystretch}{3em} % prevent overfull lines
\providecommand{\tightlist}{%
  \setlength{\itemsep}{0pt}\setlength{\parskip}{0pt}}
\setcounter{secnumdepth}{-\maxdimen} % remove section numbering
\ifLuaTeX
  \usepackage{selnolig}  % disable illegal ligatures
\fi
\IfFileExists{bookmark.sty}{\usepackage{bookmark}}{\usepackage{hyperref}}
\IfFileExists{xurl.sty}{\usepackage{xurl}}{} % add URL line breaks if available
\urlstyle{same}
\hypersetup{
  pdftitle={Challenge-6},
  pdfauthor={LIU YINGZHE},
  hidelinks,
  pdfcreator={LaTeX via pandoc}}

\title{Challenge-6}
\author{LIU YINGZHE}
\date{2023-09-18}

\begin{document}
\maketitle

\hypertarget{questions}{%
\subsection{Questions}\label{questions}}

\hypertarget{question-1-countdown-blastoff-while-loop}{%
\paragraph{Question-1: Countdown Blastoff (While
Loop)}\label{question-1-countdown-blastoff-while-loop}}

Create a program that simulates a rocket launch countdown using a while
loop. Start from 10 and countdown to ``Blastoff!'' with a one-second
delay between each countdown number. Print a message when the rocket
launches.

\textbf{Hint:} You may want to use \texttt{cat} command to print the
countdown and \texttt{Sys.sleep} for incorporating the delay

\textbf{Output preview:} Here is how the countdown could look like

\begin{Shaded}
\begin{Highlighting}[]
\NormalTok{knitr}\SpecialCharTok{::}\FunctionTok{include\_graphics}\NormalTok{(}\StringTok{"C:/Users/angel/Downloads/10.webp"}\NormalTok{)}
\end{Highlighting}
\end{Shaded}

\includegraphics[width=600px,height=200px]{../../../../Downloads/10}

\textbf{Solutions:}

\begin{Shaded}
\begin{Highlighting}[]
\CommentTok{\# Enter code here}
\NormalTok{countdown }\OtherTok{\textless{}{-}} \ControlFlowTok{function}\NormalTok{() \{}
\NormalTok{  count }\OtherTok{\textless{}{-}} \DecValTok{10}
  
  \ControlFlowTok{while}\NormalTok{ (count }\SpecialCharTok{\textgreater{}=} \DecValTok{1}\NormalTok{) \{}
    \FunctionTok{cat}\NormalTok{(count, }\StringTok{"}\SpecialCharTok{\textbackslash{}n}\StringTok{"}\NormalTok{)}
    \FunctionTok{Sys.sleep}\NormalTok{(}\DecValTok{1}\NormalTok{)  }\CommentTok{\# Delay for one second}
\NormalTok{    count }\OtherTok{\textless{}{-}}\NormalTok{ count }\SpecialCharTok{{-}} \DecValTok{1}
\NormalTok{  \}}
  
  \FunctionTok{cat}\NormalTok{(}\StringTok{"Blastoff!}\SpecialCharTok{\textbackslash{}n}\StringTok{Rocket has launched!}\SpecialCharTok{\textbackslash{}n}\StringTok{"}\NormalTok{)}
\NormalTok{\}}

\FunctionTok{countdown}\NormalTok{()}
\end{Highlighting}
\end{Shaded}

\begin{verbatim}
## 10 
## 9 
## 8 
## 7 
## 6 
## 5 
## 4 
## 3 
## 2 
## 1 
## Blastoff!
## Rocket has launched!
\end{verbatim}

\begin{Shaded}
\begin{Highlighting}[]
\NormalTok{knitr}\SpecialCharTok{::}\FunctionTok{include\_graphics}\NormalTok{(}\StringTok{"C:/Users/angel/Downloads/garrett{-}reisman{-}nasa{-}hero.webp"}\NormalTok{)}
\end{Highlighting}
\end{Shaded}

\includegraphics[width=600px,height=200px]{../../../../Downloads/garrett-reisman-nasa-hero}

\hypertarget{question-2-word-reverser-for-loop}{%
\paragraph{Question-2: Word Reverser (for
Loop)}\label{question-2-word-reverser-for-loop}}

Develop a program that takes a user-entered word and uses a while loop
to print the word's characters in reverse order. For example, if the
user enters ``hello,'' the program should print ``olleh.''

\textbf{Hint:} You may want to use \texttt{substr} command to access
each character of the input word, and \texttt{paste} command to join the
reversed letters one at a time

\textbf{Solutions:}

\begin{Shaded}
\begin{Highlighting}[]
\CommentTok{\# Enter code here}
\NormalTok{reverse\_word }\OtherTok{\textless{}{-}} \ControlFlowTok{function}\NormalTok{() \{}
\NormalTok{  word }\OtherTok{\textless{}{-}} \StringTok{"nuggets"}
\NormalTok{  reversed }\OtherTok{\textless{}{-}} \StringTok{""}
\NormalTok{  i }\OtherTok{\textless{}{-}} \FunctionTok{nchar}\NormalTok{(word)}
  
  \ControlFlowTok{while}\NormalTok{ (i }\SpecialCharTok{\textgreater{}} \DecValTok{0}\NormalTok{) \{}
\NormalTok{    reversed }\OtherTok{\textless{}{-}} \FunctionTok{paste}\NormalTok{(reversed, }\FunctionTok{substr}\NormalTok{(word, i, i), }\AttributeTok{sep =} \StringTok{""}\NormalTok{)}
\NormalTok{    i }\OtherTok{\textless{}{-}}\NormalTok{ i }\SpecialCharTok{{-}} \DecValTok{1}
\NormalTok{  \}}
  
  \FunctionTok{cat}\NormalTok{(}\StringTok{"Reversed word:"}\NormalTok{, reversed, }\StringTok{"}\SpecialCharTok{\textbackslash{}n}\StringTok{"}\NormalTok{)}
\NormalTok{\}}

\FunctionTok{reverse\_word}\NormalTok{()}
\end{Highlighting}
\end{Shaded}

\begin{verbatim}
## Reversed word: steggun
\end{verbatim}

\end{document}
